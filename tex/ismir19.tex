\newcommand{\img}{imgs/ismir19}

In order to apply the Radford et al. \cite{radford_2017} method to compose music with
sentiment, we also need a dataset of MIDI files to train the LSTM and another one
to train the logistic regression. There are many good datasets of music in MIDI format
in the literature. However, to the best of our knowledge, none are labelled according to sentiment. Thus, we created a new
dataset called VGMIDI which is composed of 823 pieces extracted
from video game soundtracks in MIDI format. We choose video game soundtracks
because they are normally composed to keep the player in a certain affective state
and thus they are less subjective pieces. All the pieces are piano
arrangements of the soundtracks and they vary in length  from 26 seconds to 3 minutes.
Among these pieces, 95 are annotated according to a 2-dimensional model
that represents emotion using a valence-arousal pair. Valence indicates
positive versus negative emotion, and arousal indicates emotional intensity \cite{Soleymani_2013}.

We use this valence-arousal model because it allows continuous annotation of music
and because of its flexibility---one can directly map a valence-arousal (v-a) pair to
a multiclass (happy, sad, surprise, etc) or a binary (positive/negative)
model. Thus, the same set of labelled data permits the investigation of affective algorithmic music composition as both a classification
(multiclass and/or binary) and as a regression problem. The valence-arousal model
is also one of the most common dimensional models used to label emotion in music
\cite{Soleymani_2013}.

Annotating a piece according to the v-a model consists of continuously
listening to the piece and deciding what valence-arousal pair best represents the emotion
of that piece in each moment, producing a time-series of v-a pairs.
This task is subjective, hence there is no single ``correct'' time-series for a given
piece. Thus, we decided to label the pieces by asking several human subjects
to listen to the pieces and then considering the average time-series as the ground truth.
This process was conducted online via Amazon Mechanical Turk, where each
piece was annotated by 30 subjects using a web-based tool we designed
specifically for this task. Each subject annotated 2 pieces out of 95,
and got rewarded USD \$0.50 for performing this task.

\section{Annotation Tool and Data Collection}
\label{sec:data_collection}

The tool we designed to annotate the video game soundtracks in MIDI format
is composed of five steps, each one being a single web-page.
These steps are
based on the methodology proposed by Soleymani et al. \cite{Soleymani_2013} for annotating
music pieces in audio waveform. First, participants are introduced to the annotation
task with a short description explaining the goal of the task and how long it
should take in average. Second, they are presented to the definitions of valence and
arousal. In the same page, they are asked to play two short pieces and indicate
whether arousal and valence are increasing or decreasing. Moreover, we ask the
annotators to write two to three sentences describing the short pieces they
listened to. This page is intended to measure their understanding of the valence-arousal model
and willingness to perform the task.  Third, a video tutorial was
made available to the annotators explaining how to use the
annotation tool. Fourth, annotators are exposed to the main
annotation page.

This main page has two phases: calibration and annotation.
In the calibration phase, annotators listen to the first 15 seconds of the
piece in order to get used to it and to define the starting
point of the annotation circle. In the annotation phase they
listen to the piece from beginning to end and label it using the annotation circle, which
starts at the point defined during the calibration phase.
Figure \ref{fig:annotation_main} shows the annotation interface
for valence and arousal, where annotators click and hold the circle (with the play icon)
inside the v-a model (outer circle) indicating the current emotion
of the piece. In order to maximize annotators’ engagement in the task, the piece
is only played while they maintain a click on the play circle.
In addition, basic instructions on how to use the tool are showed to the participants
along with the definitions of valence and arousal. A progression bar is also
showed to the annotators so they know how far they are from completing each phase.
This last step (calibration and annotation) is repeated for a second piece.
All of the pieces the annotators listened to are MIDI files synthesized with the
``Yamaha C5 Grand" soundfont. Finally, after the main annotation step, participants
provide demographic information including gender, age, location (country), musicianship
experience and whether they previously knew the pieces they annotated.

\begin{figure}
 \centering
 \includegraphics[width=\columnwidth]{\img/annotation_tool.png}
 \caption{Screenshot of the annotation tool.}
 \label{fig:annotation_main}
\end{figure}

\section{Data Analysis}
\label{sec:data_analysys}

The annotation task was performed by 1425 annotators, where 55\% are female and 42\%
are male. The other 3\% classified themselves as transgender female,
transgender male, genderqueer or choose not to disclose their gender.
All annotators are from the United States and have an average age of approximately 31 years.
Musicianship experience was assessed using a 5-point Likert scale where 1 means
``I've never studied music theory or practice'' and 5 means ``I have an undergraduate
degree in music''. The average musicianship experience is 2.28. They spent on average
12 minutes and 6 seconds to annotate the 2 pieces.

The data collection process provides a time series of valence-arousal values for
each piece, however to create a music sentiment dataset we only need the valence
dimension, which encodes negative and positive sentiment.
Thus, we consider that each piece has 30 time-series of valence values. The
annotation of each piece was preprocessed,
summarized into one time-series and split into ``phrases''
of same sentiment. The preprocessing is intended to remove
noise caused by subjects performing the
task randomly to get the reward as fast as possible.
The data was preprocessed by smoothing each annotation with moving average
and clustering all 30 time-series into 3 clusters (positive, negative and noise)
according to the dynamic time-warping distance metric.

We consider the cluster with the highest variance to be noise cluster and so we
discard it. The cluster with more time series among the two remaining ones is
then selected and summarized by the mean of its time series.
We split this mean into several segments with the same sentiment.
This is performed by splitting the
mean at all the points where the valence changes
from positive to negative or vice-versa. Thus,
all chunks with negative valence are considered phrases
with negative sentiment and the ones with positive valence are positive
phrases. Figure \ref{fig:clustering} shows an example of this three-steps process
performed on a piece.

\begin{figure}
 \centering
 \includegraphics[width=\columnwidth]{\img/clustering.png}
 \caption{Data analysis process used to define the final label of the phrases of a piece. }
 \label{fig:clustering}
\end{figure}

All the phrases that had no notes (i.e. silence phrases)
were removed. This process created a total of 966 phrases: 599 positive
and 367 negative. Table \ref{tab:dataset} shows a snippet of the dataset.

\begin{table}[!h]
 \begin{center}
 \begin{tabular}{ccc}
  \hline
  \textbf{Label} & \textbf{Piece ID} & \textbf{MIDI Phrase} \\
  \hline
     0 & 14 & piece14\_phrase6.mid \\
     0 & 14 & piece14\_phrase7.mid \\
     1 & 15 & piece15\_phrase0.mid \\
  \hline
 \end{tabular}
\end{center}
 \caption{Snippet of the VGMIDI labeled phrases.}
 \label{tab:dataset}
\end{table}

% Deep Learning has recently achieved incredible results in music composition,
% but one still can't guide the proposed models to generate compositions
% with a given goal. This paper is motivated by the application of soundtrack
% generation and so it focus on the controlling emotions of the generated pieces.
We propose a Deep Learning method for affective algorithmic composition that
can be controlled to generate music with a given sentiment. This method is based on the work of
Radford et al. \cite{radford_2017} which generates product reviews (in textual form) with sentiment.
Radford et al. \cite{radford_2017} used a single-layer multiplicative
long short-term memory (mLSTM) network \cite{krause2017} with 4096 units to process
text as a sequence of UTF-8 encoded bytes (character-based language modeling).
% The architecture of the network is illustrated in Figure \ref{fig:mlstm}.
For each byte, the model updates its hidden state of the mLSTM and predicts a
probability distribution over the next possible byte.
% The hidden state of the model
% serves as an online summary of the sequence which encodes all information
% the model has learned to preserve that is relevant to predicting the future
% bytes of the sequence.

% \begin{center}
% \begin{tikzpicture}
%     \draw[thick,->] (0,2) node[anchor=south] {input} -- (1,2);

%     \draw (1,1) -- (1.5,1) -- (1.5,3) -- (1,3)  -- (1,1);

%     \draw[thick,->] (1.5,2) node [anchor=south, rotate=90] {Embed.} -- (2.5,2);

%     \draw (2.5,0) -- (3,0) -- (3,4) -- (2.5,4) -- (2.5,0);

%     \draw[thick,->] (3,2) node [anchor=south, rotate=90] {mLSTM} -- (4,2);

%     \draw (4,1) -- (4.5,1) -- (4.5,3) -- (4,3) -- (4,1);

%     \draw[thick,->] (4.5,2) node [anchor=south, rotate=90] {Softmax} -- (5.5,2) node[anchor=south] {output};
% \end{tikzpicture}
% \captionof{figure}{Architecture of the mLSTM used to process text.}
% \label{fig:mlstm}
% \end{center}

This mLSTM was trained on the Amazon product review dataset, which contains over 82 million
product reviews from May 1996 to July 2014 amounting to over 38 billion
training bytes \cite{He2016}. Radford et al. \cite{radford_2017} used the trained mLSTM
to encode sentences from four different Sentiment Analysis datasets.
The encoding is performed by initializing the the states to zeros and
processing the sequence character-by-character. The final hidden states of
the mLSTM are used as a feature representation. With the encoded
datasets, Radford et al. \cite{radford_2017} trained a simple logistic
regression classifier with L1 regularization and outperformed
the state-of-the-art methods at the time using 30-100x fewer
labeled examples.

By inspecting the relative contributions of features on various datasets,
Radford et al. \cite{radford_2017} discovered a single unit within the
mLSTM that directly corresponded to sentiment. Because the mLSTM was
trained as a generative model, one can simply set the value of the
sentiment unit to be positive or negative and the model generates
corresponding positive or negative reviews.

\section{Data Representation}
\label{sec:representation}

We use the same combination of mLSTM and logistic
regression to compose music with sentiment. To do this, we
treat the music composition problem as a language modeling problem.
Instead of characters, we represent a music piece as a
sequence of words and punctuation marks from a vocabulary that represents
events retrieved from the MIDI file. Sentiment is perceived in
music due to several features such as melody, harmony, tempo, timbre,
etc \cite{kim2010music}. Our data representation attempts to encode a large part
of these features\footnote{Constrained by the features one can extract from MIDI data.}
using a small set of words:

\begin{itemize}
    \item ``n\_[pitch]'': play note with given pitch number: any integer from 0 to 127.
    \item ``d\_[duration]\_[dots]'': change the duration of the following notes to a given
    duration type with a given amount of dots. Types are breve, whole, half, quarter,
    eighth, 16th and 32nd. Dots can be any integer from 0 to 3.
    \item ``v\_[velocity]'': change the velocity of the following  notes to a given velocity (loudness) number. Velocity is discretized in
    bins of size 4, so it can be any integer in the set $V = {4, 8, 12, \dots, 128}$.
    \item ``t\_[tempo]'': change the tempo of the piece to a given tempo in bpm. Tempo is also discretized in bins of size 4, so it can be any integer in the set $T = {24, 28, 32, \dots, 160}$.
    \item ``.'': end of time step. Each time step is one sixteenth note long.
    \item ``\textbackslash n'': end of piece.
\end{itemize}

For example, Figure \ref{fig:enc_ex} shows the encoding of the first two time steps of the
first measure of the Legend of Zelda - Ocarina of Time's Prelude of Light.
The first time step sets the tempo to 120bpm, the velocity of the following notes to
76 and plays the D Major Triad for the duration of a whole note. The second time step sets the velocity to 84 and plays a dotted quarter A5 note. The total size of this vocabulary is 225 and it represents both the composition and performance elements of a piece (timing and dynamics).

\begin{figure}
 \centering
 \includegraphics[width=\columnwidth]{\img/encoding.pdf}
\begin{spverbatim}
t_120 v_76 d_whole_0 n_50 n_54 n_57
v_92 d_eighth n_86 . . v_84 d_quarter_1 n_81 . .
\end{spverbatim}

 \caption{A short example piece encoded using our proposed representation. The encoding represents the first two time steps of the shown measure.}
 \label{fig:enc_ex}
\end{figure}


\section{Sentiment Analysis Evaluation}

To evaluate the sentiment classification accuracy of our method (generative mLSTM +
logistic regression), we compare it to a baseline method which is a
traditional classification mLSTM trained in a supervised way. Our method uses
unlabelled MIDI pieces to train a generative mLSTM to predict the next word in a
sequence. An additional logistic regression uses the hidden states of the generative
mLSTM to encode the labelled MIDI phrases and then predict sentiment.
The baseline method uses only labelled MIDI phrases to train a
classification mLSTM to predict the sentiment for the phrase.
% It is representative of existing sentiment prediction models. Both approaches are described more fully below.

% In the generative mLSTM + logistic regression approach, we first use the unlabelled MIDI pieces for training the
% generative (predicting the next word in a sequence) mLSTM.
The unlabelled pieces used to train the generative mLSTM  were
transformed in order to create additional training examples,
following the methodology of Oore et al.
\cite{oore2017learning}. The transformations consist of time-stretching (making
each piece up to 5\% faster or slower) and transposition (raising or lowering
the pitch of each piece by up to a major third). We then encoded all these
pieces and transformations according to our word-based representation (see
Section \ref{sec:representation}). Finally, the encoded pieces were shuffled
and 90\% of them were used for training and 10\% for testing. The training set was divided into 3 shards of similar size (approximately 18500 pieces each -- 325MB) and the testing set was combined into 1 shard (approximately 5800 pieces -- 95MB).

We trained the generative mLSTM with 6 different sizes (number of neurons
in the mLSTM layer): 128, 256, 512, 1024, 2048 and 4096. For each size,
the generative mLSTM was trained for 4 epochs using the 3 training
shards. Weights were updated with the Adam optimizer after processing
sequences of 256 words on mini-batches of size 32. The mLSTM hidden
and cell states were initialized to zero at the beginning of each
shard. They were also persisted across updates to simulate
full-backpropagation and allow for the forward propagation of
information outside of a given sequence \cite{radford_2017}. Each sequence is
processed by an embedding layer (which is trained together with
the mLSTM layer) with 64 neurons before passing through the mLSTM layer.
The learning rate was set to $5*10^{-6}$ at the beginning and decayed linearly
(after each epoch) to zero over the course of training.

We evaluated each variation of the generative mLSTM with a
forward pass on
test shard using mini-batches of size 32. Table
\ref{tab:gen_anal} shows the average\footnote{Each mini-batch
reports one loss.} cross entropy loss
for each variation of the generative mLSTM.

\begin{table}[!h]
 \begin{center}
 \begin{tabular}{cc}
  \hline
  \textbf{mLSTM Neurons} & \textbf{Average Cross Entropy Loss}\\ \hline
  128 & 1.80   \\
  256  & 1.61  \\
  512  & 1.41  \\
  1024 & 1.25  \\
  2048 & 1.15  \\
  4096 & 1.11  \\ \hline
 \end{tabular}
\end{center}
\caption{Average cross entropy loss of the generative mLSTM with different amount of neurons.}
 \label{tab:gen_anal}
\end{table}

The average cross entropy loss decreases as the size of the mLSTM increases, reaching the best result (loss 1.11) when size is equal to 4096. Thus, we used the variation with 4096 neurons to proceed with the sentiment classification experiments.

Following the methodology of Radford et al. \cite{radford_2017}, we re-encoded
each of the 966 labelled phrases using the final cell states (a 4096 dimension vector)
of the trained generative mLSTM-4096. The states are calculated by initializing them to zero
and processing the phrase word-by-word. We plug a logistic regression into the mLSTM-4096
to turn it into a sentiment classifier. This logistic regression model was trained with
regularization ``L1'' to shrink the least important of the 4096 feature weights to zero.
This ends up highlighting the generative mLSTM neurons that contain most of the sentiment
signal.

We compared this generative mLSTM + logistic regression approach against
our baseline, the supervised mLSTM. This is an mLSTM with exactly the same architecture
and size of the generative version, but trained in a fully supervised way.
To train this supervised mLSTM, we used the word-based representation of the phrases, but we
padded each phrase with silence (the symbol ``.'') in order to equalize their length.
Training parameters (learning rate and decay, epochs, batch size, etc) were the same ones
of the the generative mLSTM. It is important to notice that in this case the mini-batches are
formed of 32 labelled phrases and not words. We evaluate both methods using a 10-fold cross
validation approach, where the test folds have no phrases that appear in the training folds.
Table \ref{tab:sent_anal} shows the sentiment classification accuracy of both approaches.

\begin{table}[!h]
 \begin{center}
 \begin{tabular}{lc}
  \hline
  \textbf{Method} & \textbf{Test Accuracy} \\ \hline
  Gen. mLSTM-4096 + Log. Reg. & 89.83 $\pm$ 3.14\\
  Sup. mLSTM-4096             & 60.35 $\pm$ 3.52 \\
  \hline
 \end{tabular}
\end{center}
\caption{Average (10-fold cross validation) sentiment classification accuracy of both generative (with logistic regression) and supervised mLSTMs.}
 \label{tab:sent_anal}
\end{table}

The generative mLSTM with logistic regression achieved an
accuracy of 89.83\%, outperforming the supervised mLSTM by 29.48\%. The supervised
mLSTM  accuracy of 60.35\% suggests that the amount of labelled data (966 phrases)
was not enough to learn a good mapping between phrases and sentiment.
The accuracy of our method shows that the generative mLSTM is capable of learning, in
an unsupervised way, a good representation of sentiment in symbolic music.

This is an important result, for two reasons. First, since the higher accuracy of generative
mLSTM is derived from using unlabeled data, it will be easier to improve this over time using additional
(less expensive) unlabeled data, instead of the supervised mLSTM approach which requires additional (expensive)
labeled data. Second, because the generative mLSTM was trained to predict the next word in a sequence, it can
be used as a music generator. Since it is combined with a sentiment predictor, it opens up the possibility of
generating music consistent with a desired sentiment. We explore this idea in the following section.

\section{Generative Evaluation}

To control the sentiment of the music generated by our mLSTM, we find the
subset of neurons that contain the sentiment signal by exploring the weights
of the trained logistic regression model. Since each of the 10 generative models derived
from the 10 fold splits in Table \ref{tab:sent_anal} are themselves a full model, we use
the model with the highest accuracy.
%For this analysis, we used the experiment
%of the 10-fold cross validation with higher accuracy.
As shown in Figure
\ref{fig:final_weights}, the logistic regression trained with regularization ``L1''
uses 161 neurons out of 4096. Unlike the results of Radford et al. \cite{radford_2017},
we don't have one single neuron that stores most of the sentiment signal. Instead,
we have many neurons contributing in a more balanced way. Therefore, we can't simply
change the values of one neuron to control the sentiment of the output music.

\begin{figure}[!h]
 \centering
 \includegraphics[width=\columnwidth]{\img/weights.png}
 \caption{Weights of 161 L1 neurons. Note multiple prominent positive and negative neurons.}
 \label{fig:final_weights}
\end{figure}

We used a Genetic Algorithm (GA) to optimize the weights of the
161 L1 neurons in order to lead our mLSTM to generate only positive
or negative pieces. Each individual in the population of this GA has $161$
real-valued genes representing a small noise to be added to the weights of the $161$ L1 neurons. The fitness of an individual is computed by (i) adding the genes of the individual to the weights (vector addition) of the $161$ L1 neurons of the generative mLSTM, (ii)
generating $P$ pieces with this mLSTM, (iii) using the logistic regression model to predict
these $P$ generated pieces and (iv) calculating the mean squared error of the $P$
predictions given a desired sentiment $s \in S = \{0, 1\}$.

The GA starts with a random population of size 100 where each gene of
each individual is an uniformly sampled random number $-2 \leq r \leq 2$.
For each generation, the GA (i) evaluates the current population,
(ii) selects 100 parents via a roulette wheel with elitism, (iii) recombines the
parents (crossover) taking the average of their genes and (iv) mutates each
new recombined individual (new offspring) by randomly setting each gene to
an uniformly sampled random number $-2 \leq r \leq 2$.

We performed two independent executions of this GA,
one to optimize the mLSTM for generating positive pieces
and another one for negative pieces. Each execution optimized the
individuals during 100 epochs with crossover rate of 95\% and
mutation rate of 10\%. To calculate the fitness of each individual, we generated $P$=30 pieces with 256 words each, starting with the symbol ``.'' (end of time step).
The optimization for positive and negative generation resulted in best individuals with
fitness $0.16$ and $0.33$, respectively. This means that if we
add the genes of the best individual of the final population to the weights of the generative mLSTM, we generate positive pieces
with 84\% accuracy and negative pieces with 67\% accuracy.

After these two optimization processes, the genes of the
best final individual of the positive optimization were
added to the  weights of the 161 L1 neurons of the trained
generative mLSTM. We then generated 30 pieces with 1000
words starting with the symbol ``.'' (end of time step) and
randomly selected 3 of them. The same process was repeated
using the genes of the best final individual of the
negative execution. We asked annotators to label this 6
generated pieces via Amazon MTurk, using the the same
methodology described in Section \ref{sec:data_collection}.
Figure \ref{fig:generated_eval} shows the average valence per measure of each of the generated pieces.

\begin{figure}[!h]
 \includegraphics[width=\columnwidth]{\img/means_pos.png}
 \includegraphics[width=\columnwidth]{\img/means_neg.png}
 \caption{Average valence of the 6 generated pieces, as determined by human annotators.
 %Note that the averages are calculated after clustering the annotations and selecting the cluster
 with least variance.}
 \label{fig:generated_eval}
\end{figure}

We observe that the human annotators agreed that the three positive generated pieces are indeed positive.
The generated negative pieces are more ambiguous, having both negative and positive measures.  However, as a whole the negative pieces have lower valence
than the positive ones. This suggests that
the best negative individual (with fitness $0.33$) encountered by the GA
wasn't good enough to control the mLSTM to generate complete negative pieces. Moreover, the challenge to optimize the L1 neurons suggests that there are more positive pieces than negative ones in the 3 shards used to train the generative mLSTM.

\section{Conclusions}

This paper presented a generative mLSTM that can be controlled to generate symbolic music with a given sentiment. The mLSTM is controlled by optimizing the weights
of specific neurons that are responsible for the sentiment signal. Such neurons are
found plugging a Logistic Regression to the mLSTM and training the Logistic Regression
to classify sentiment of symbolic music encoded with the mLSTM hidden states. We evaluated
this model both as a generator and as a sentiment classifier. Results showed that
our model obtained good classification accuracy, outperforming a equivalent LSTM
trained in a fully supervised way. Moreover, a user study showed that humans agree
that our model can generate positive and negative music, with the caveat that the
negative pieces are more ambiguous.

In the future, we plan to improve our model to generate less ambiguous negative pieces. Another
future work consists of expanding the model to generate music with a given
emotion (e.g. happy, sad, suspenseful, etc.) as well as with a given valence-arousal pair
(real numbers). We also plan to use this model to compose soundtracks in real-time for
oral storytelling experiences  \cite{padovani2017bardo}.

% There is less agreement on the negative pieces. However, as a whole the negative pieces have lower valence
% than the positive ones. We noticed that the individual v-a traces for the negative pieces had greater
% variation, and both clusters have high variance, indicating less agreement among the raters for these pieces.
% It is unclear why the negative pieces had so much more variation than the positive ones.

%A drawback of our use of MTurk is an inability to directly question people about their rating choices to determine why they made specific ratings. We will pursue this in future work.

%We conjecture that it might be due to the lack of a defined ending to the pieces. One piece had a high tempo, which might have been interpreted as positive valence.
